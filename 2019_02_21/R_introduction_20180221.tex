\documentclass[]{article}
\usepackage{lmodern}
\usepackage{amssymb,amsmath}
\usepackage{ifxetex,ifluatex}
\usepackage{fixltx2e} % provides \textsubscript
\ifnum 0\ifxetex 1\fi\ifluatex 1\fi=0 % if pdftex
  \usepackage[T1]{fontenc}
  \usepackage[utf8]{inputenc}
\else % if luatex or xelatex
  \ifxetex
    \usepackage{mathspec}
  \else
    \usepackage{fontspec}
  \fi
  \defaultfontfeatures{Ligatures=TeX,Scale=MatchLowercase}
\fi
% use upquote if available, for straight quotes in verbatim environments
\IfFileExists{upquote.sty}{\usepackage{upquote}}{}
% use microtype if available
\IfFileExists{microtype.sty}{%
\usepackage{microtype}
\UseMicrotypeSet[protrusion]{basicmath} % disable protrusion for tt fonts
}{}
\usepackage[margin=1in]{geometry}
\usepackage{hyperref}
\hypersetup{unicode=true,
            pdftitle={Bioinformatics - R Introduction},
            pdfauthor={Wei-Hao, Lee},
            pdfborder={0 0 0},
            breaklinks=true}
\urlstyle{same}  % don't use monospace font for urls
\usepackage{color}
\usepackage{fancyvrb}
\newcommand{\VerbBar}{|}
\newcommand{\VERB}{\Verb[commandchars=\\\{\}]}
\DefineVerbatimEnvironment{Highlighting}{Verbatim}{commandchars=\\\{\}}
% Add ',fontsize=\small' for more characters per line
\usepackage{framed}
\definecolor{shadecolor}{RGB}{248,248,248}
\newenvironment{Shaded}{\begin{snugshade}}{\end{snugshade}}
\newcommand{\KeywordTok}[1]{\textcolor[rgb]{0.13,0.29,0.53}{\textbf{{#1}}}}
\newcommand{\DataTypeTok}[1]{\textcolor[rgb]{0.13,0.29,0.53}{{#1}}}
\newcommand{\DecValTok}[1]{\textcolor[rgb]{0.00,0.00,0.81}{{#1}}}
\newcommand{\BaseNTok}[1]{\textcolor[rgb]{0.00,0.00,0.81}{{#1}}}
\newcommand{\FloatTok}[1]{\textcolor[rgb]{0.00,0.00,0.81}{{#1}}}
\newcommand{\ConstantTok}[1]{\textcolor[rgb]{0.00,0.00,0.00}{{#1}}}
\newcommand{\CharTok}[1]{\textcolor[rgb]{0.31,0.60,0.02}{{#1}}}
\newcommand{\SpecialCharTok}[1]{\textcolor[rgb]{0.00,0.00,0.00}{{#1}}}
\newcommand{\StringTok}[1]{\textcolor[rgb]{0.31,0.60,0.02}{{#1}}}
\newcommand{\VerbatimStringTok}[1]{\textcolor[rgb]{0.31,0.60,0.02}{{#1}}}
\newcommand{\SpecialStringTok}[1]{\textcolor[rgb]{0.31,0.60,0.02}{{#1}}}
\newcommand{\ImportTok}[1]{{#1}}
\newcommand{\CommentTok}[1]{\textcolor[rgb]{0.56,0.35,0.01}{\textit{{#1}}}}
\newcommand{\DocumentationTok}[1]{\textcolor[rgb]{0.56,0.35,0.01}{\textbf{\textit{{#1}}}}}
\newcommand{\AnnotationTok}[1]{\textcolor[rgb]{0.56,0.35,0.01}{\textbf{\textit{{#1}}}}}
\newcommand{\CommentVarTok}[1]{\textcolor[rgb]{0.56,0.35,0.01}{\textbf{\textit{{#1}}}}}
\newcommand{\OtherTok}[1]{\textcolor[rgb]{0.56,0.35,0.01}{{#1}}}
\newcommand{\FunctionTok}[1]{\textcolor[rgb]{0.00,0.00,0.00}{{#1}}}
\newcommand{\VariableTok}[1]{\textcolor[rgb]{0.00,0.00,0.00}{{#1}}}
\newcommand{\ControlFlowTok}[1]{\textcolor[rgb]{0.13,0.29,0.53}{\textbf{{#1}}}}
\newcommand{\OperatorTok}[1]{\textcolor[rgb]{0.81,0.36,0.00}{\textbf{{#1}}}}
\newcommand{\BuiltInTok}[1]{{#1}}
\newcommand{\ExtensionTok}[1]{{#1}}
\newcommand{\PreprocessorTok}[1]{\textcolor[rgb]{0.56,0.35,0.01}{\textit{{#1}}}}
\newcommand{\AttributeTok}[1]{\textcolor[rgb]{0.77,0.63,0.00}{{#1}}}
\newcommand{\RegionMarkerTok}[1]{{#1}}
\newcommand{\InformationTok}[1]{\textcolor[rgb]{0.56,0.35,0.01}{\textbf{\textit{{#1}}}}}
\newcommand{\WarningTok}[1]{\textcolor[rgb]{0.56,0.35,0.01}{\textbf{\textit{{#1}}}}}
\newcommand{\AlertTok}[1]{\textcolor[rgb]{0.94,0.16,0.16}{{#1}}}
\newcommand{\ErrorTok}[1]{\textcolor[rgb]{0.64,0.00,0.00}{\textbf{{#1}}}}
\newcommand{\NormalTok}[1]{{#1}}
\usepackage{graphicx,grffile}
\makeatletter
\def\maxwidth{\ifdim\Gin@nat@width>\linewidth\linewidth\else\Gin@nat@width\fi}
\def\maxheight{\ifdim\Gin@nat@height>\textheight\textheight\else\Gin@nat@height\fi}
\makeatother
% Scale images if necessary, so that they will not overflow the page
% margins by default, and it is still possible to overwrite the defaults
% using explicit options in \includegraphics[width, height, ...]{}
\setkeys{Gin}{width=\maxwidth,height=\maxheight,keepaspectratio}
\IfFileExists{parskip.sty}{%
\usepackage{parskip}
}{% else
\setlength{\parindent}{0pt}
\setlength{\parskip}{6pt plus 2pt minus 1pt}
}
\setlength{\emergencystretch}{3em}  % prevent overfull lines
\providecommand{\tightlist}{%
  \setlength{\itemsep}{0pt}\setlength{\parskip}{0pt}}
\setcounter{secnumdepth}{0}
% Redefines (sub)paragraphs to behave more like sections
\ifx\paragraph\undefined\else
\let\oldparagraph\paragraph
\renewcommand{\paragraph}[1]{\oldparagraph{#1}\mbox{}}
\fi
\ifx\subparagraph\undefined\else
\let\oldsubparagraph\subparagraph
\renewcommand{\subparagraph}[1]{\oldsubparagraph{#1}\mbox{}}
\fi

%%% Use protect on footnotes to avoid problems with footnotes in titles
\let\rmarkdownfootnote\footnote%
\def\footnote{\protect\rmarkdownfootnote}

%%% Change title format to be more compact
\usepackage{titling}

% Create subtitle command for use in maketitle
\newcommand{\subtitle}[1]{
  \posttitle{
    \begin{center}\large#1\end{center}
    }
}

\setlength{\droptitle}{-2em}

  \title{Bioinformatics - R Introduction}
    \pretitle{\vspace{\droptitle}\centering\huge}
  \posttitle{\par}
    \author{Wei-Hao, Lee}
    \preauthor{\centering\large\emph}
  \postauthor{\par}
      \predate{\centering\large\emph}
  \postdate{\par}
    \date{2/21/2019}


\begin{document}
\maketitle

\section{R 基礎教學}\label{r-}

\subsection{目錄}

\begin{enumerate}
\def\labelenumi{\arabic{enumi}.}
\tightlist
\item
  變數 - Variable :

  \begin{itemize}
  \tightlist
  \item
    變數指派
  \item
    移除變數
  \end{itemize}
\item
  運算子 - Operator :

  \begin{itemize}
  \tightlist
  \item
    Arithmetic Operator
  \item
    Logical Operator
  \end{itemize}
\item
  變數類型與資料型別 :

  \begin{itemize}
  \tightlist
  \item
    變數類型

    \begin{itemize}
    \tightlist
    \item
      Numeric
    \item
      Integer
    \item
      Logical
    \item
      Character
    \end{itemize}
  \item
    資料型別

    \begin{itemize}
    \tightlist
    \item
      One Dimension

      \begin{itemize}
      \tightlist
      \item
        Vector
      \item
        Factor
      \end{itemize}
    \item
      Two Dimension (Next Class)

      \begin{itemize}
      \tightlist
      \item
        Matrix
      \item
        Dataframe
      \end{itemize}
    \item
      Multi-Dimension (Next Class)

      \begin{itemize}
      \tightlist
      \item
        List
      \end{itemize}
    \end{itemize}
  \end{itemize}
\end{enumerate}

\subsubsection{1. 變數 - Variable}\label{---variable}

變數主要用來存取任何合法資料,使用者可以方便地使用變數名稱來獲取或者使用資料。

\begin{Shaded}
\begin{Highlighting}[]
\CommentTok{# 將"偉豪"這個資料存入變數 teacher_assistant中}
\NormalTok{teacher_assistant <-}\StringTok{ "偉豪"}

\CommentTok{# 我們就可以利用 teacher_assistant 這個變數來獲取裡面的資料}
\KeywordTok{print}\NormalTok{(teacher_assistant)}
\end{Highlighting}
\end{Shaded}

\begin{verbatim}
## [1] "偉豪"
\end{verbatim}

\paragraph{變數指派}

在上面的程式碼中,可以看到\texttt{\textless{}-}這個符號。這個符號用於將資料\textbf{賦予
(assign)}給變數。以上述的例子,我們將\texttt{"偉豪"}這串文字賦值給變數\texttt{teacher\_assistant}中。在其他程式語言中,比較常用\texttt{=}當作賦值的符號,而R語言也有支援這種\texttt{=}賦值符號。

\begin{Shaded}
\begin{Highlighting}[]
\CommentTok{# R語言中最常使用的方法,也是最道地的用法}
\NormalTok{TA_1 <-}\StringTok{ "乃文"}

\CommentTok{# 在其他程式語言中賦值的方法,在許多時候仍可以看到。}
\NormalTok{TA_2 =}\StringTok{ "嘉琪"}

\CommentTok{# 也是賦值的方法之一,但極不推薦使用。}
\StringTok{"漢萱"} \NormalTok{->}\StringTok{ }\NormalTok{TA_3}

\KeywordTok{cat}\NormalTok{(TA_1, TA_2, TA_3)}
\end{Highlighting}
\end{Shaded}

\begin{verbatim}
## 乃文 嘉琪 漢萱
\end{verbatim}

\paragraph{移除變數}

當變數或者資料不再需要使用時,我們可以移除變數,讓這些資料不佔用電腦的記憶體。我們可以使用\texttt{rm()}這個函式將變數以及內部的資料給刪除。

\begin{Shaded}
\begin{Highlighting}[]
\NormalTok{TA_4 <-}\StringTok{ "玉潔"}
\KeywordTok{print}\NormalTok{(TA_4)}
\end{Highlighting}
\end{Shaded}

\begin{verbatim}
## [1] "玉潔"
\end{verbatim}

\begin{Shaded}
\begin{Highlighting}[]
\CommentTok{# 當變數 TA_4 不再需要使用時,可以將它刪除}
\KeywordTok{rm}\NormalTok{(TA_4)}

\CommentTok{# 若再次呼叫變數 TA_4 則會得到下方的警告}
\NormalTok{TA_4}
\end{Highlighting}
\end{Shaded}

\begin{verbatim}
## Error in eval(expr, envir, enclos): object 'TA_4' not found
\end{verbatim}

\paragraph{變數命名}

變數在命名,盡量以明確為主要的命名方式,不要以代號的形式作為變數名稱。為了程式的可讀性,我們會盡量在命名時花點心思,讓我們日後再次編輯程式時,不會感到困惑。

\begin{Shaded}
\begin{Highlighting}[]
\CommentTok{# 最不推薦的方式,更不要以a,b,c,d為順序去做命名}
\NormalTok{a <-}\StringTok{ "MAPKKK"}

\CommentTok{# 明確的指出 "TP53" 是我們主要的目標基因}
\NormalTok{target_gene <-}\StringTok{ "TP53"}
\end{Highlighting}
\end{Shaded}

\subsubsection{2. 運算子 - Operator}\label{---operator}

\paragraph{算術運算子 - Arithmetic
Operator}\label{---arithmetic-operator}

算術運算子就是我們幼稚園教的加減乘除\texttt{+}、\texttt{-}、\texttt{*}、\texttt{/}。而R語言中有提供更多的算術運算子像是次方\texttt{\^{}}
or \texttt{**}
以及獲取餘數\texttt{\%\%}。\emph{(補充說明:在不同程式語言中,他們有各自定義的運算子,即使符號相同其代表的意義也不同,所以在使用不同程式語言時不妨看看其運算子的定義)}

\begin{Shaded}
\begin{Highlighting}[]
\NormalTok{x <-}\StringTok{ }\DecValTok{4}
\NormalTok{y <-}\StringTok{ }\DecValTok{3}

\CommentTok{# 4 + 3}
\NormalTok{x +}\StringTok{ }\NormalTok{y}
\end{Highlighting}
\end{Shaded}

\begin{verbatim}
## [1] 7
\end{verbatim}

\begin{Shaded}
\begin{Highlighting}[]
\CommentTok{# 4 - 3}
\NormalTok{x -}\StringTok{ }\NormalTok{y}
\end{Highlighting}
\end{Shaded}

\begin{verbatim}
## [1] 1
\end{verbatim}

\begin{Shaded}
\begin{Highlighting}[]
\CommentTok{# 4 * 3}
\NormalTok{x *}\StringTok{ }\NormalTok{y }
\end{Highlighting}
\end{Shaded}

\begin{verbatim}
## [1] 12
\end{verbatim}

\begin{Shaded}
\begin{Highlighting}[]
\CommentTok{# 4 / 3}
\NormalTok{x /}\StringTok{ }\NormalTok{y}
\end{Highlighting}
\end{Shaded}

\begin{verbatim}
## [1] 1.333333
\end{verbatim}

\begin{Shaded}
\begin{Highlighting}[]
\CommentTok{# 4 / 3 = 3...1 取餘數 1}
\NormalTok{x %%}\StringTok{ }\NormalTok{y }
\end{Highlighting}
\end{Shaded}

\begin{verbatim}
## [1] 1
\end{verbatim}

\begin{Shaded}
\begin{Highlighting}[]
\CommentTok{# 4 * 4 + 3 * 3 = 25}
\NormalTok{x^}\DecValTok{2} \NormalTok{+}\StringTok{ }\NormalTok{y**}\DecValTok{2}
\end{Highlighting}
\end{Shaded}

\begin{verbatim}
## [1] 25
\end{verbatim}

\emph{補充說明:以次方為例,python中\texttt{\^{}}是 bitwise
operator,而非代表次方。另外在其他程式語言像Java中並沒有支援\texttt{**}
or \texttt{\^{}}等運算子}

\paragraph{邏輯運算子 - Logical Operator}\label{---logical-operator}

邏輯運算子包含大於\texttt{\textgreater{}}、小於\texttt{\textless{}}、大於等於\texttt{\textgreater{}=}、小於等於\texttt{\textless{}=}、等於\texttt{==}、不等於\texttt{!=}、交集\texttt{\&\&}、聯集\texttt{\textbar{}\textbar{}}。透過兩者數值的比較之後告訴你布林值(boolean:TRUE
or FALSE)。

\begin{Shaded}
\begin{Highlighting}[]
\NormalTok{weight_a <-}\StringTok{ }\DecValTok{100}
\NormalTok{weight_b <-}\StringTok{ }\DecValTok{20}

\CommentTok{# 100 > 20 : TRUE}
\NormalTok{weight_a >}\StringTok{ }\NormalTok{weight_b}
\end{Highlighting}
\end{Shaded}

\begin{verbatim}
## [1] TRUE
\end{verbatim}

\begin{Shaded}
\begin{Highlighting}[]
\CommentTok{# 100 < 20: FALSE}
\NormalTok{weight_a <}\StringTok{ }\NormalTok{weight_b}
\end{Highlighting}
\end{Shaded}

\begin{verbatim}
## [1] FALSE
\end{verbatim}

\begin{Shaded}
\begin{Highlighting}[]
\CommentTok{# 100 >= 20: TRUE}
\NormalTok{weight_a >=}\StringTok{ }\NormalTok{weight_b}
\end{Highlighting}
\end{Shaded}

\begin{verbatim}
## [1] TRUE
\end{verbatim}

\begin{Shaded}
\begin{Highlighting}[]
\CommentTok{# 100 <= 20: FALSE}
\NormalTok{weight_a <=}\StringTok{ }\NormalTok{weight_b}
\end{Highlighting}
\end{Shaded}

\begin{verbatim}
## [1] FALSE
\end{verbatim}

\begin{Shaded}
\begin{Highlighting}[]
\CommentTok{# 100 == 20: FALSE}
\NormalTok{weight_a ==}\StringTok{ }\NormalTok{weight_b}
\end{Highlighting}
\end{Shaded}

\begin{verbatim}
## [1] FALSE
\end{verbatim}

\begin{Shaded}
\begin{Highlighting}[]
\CommentTok{# 100 != 20: TRUE}
\NormalTok{weight_a !=}\StringTok{ }\NormalTok{weight_b}
\end{Highlighting}
\end{Shaded}

\begin{verbatim}
## [1] TRUE
\end{verbatim}

\begin{Shaded}
\begin{Highlighting}[]
\CommentTok{# 兩個敘述都為 TRUE 時,才會回傳 TRUE}
\CommentTok{# 若其中一者為 FALSE 時,則回傳 FALSE}

\CommentTok{# 100 > 30 and 20 < 200 : TRUE}
\NormalTok{weight_a >}\StringTok{ }\DecValTok{30} \NormalTok{&&}\StringTok{ }\NormalTok{weight_b <}\StringTok{ }\DecValTok{200}
\end{Highlighting}
\end{Shaded}

\begin{verbatim}
## [1] TRUE
\end{verbatim}

\begin{Shaded}
\begin{Highlighting}[]
\CommentTok{# 100 < 30 and 20 < 200 : FALSE}
\NormalTok{weight_a <}\StringTok{ }\DecValTok{30} \NormalTok{&&}\StringTok{ }\NormalTok{weight_b <}\StringTok{ }\DecValTok{200}
\end{Highlighting}
\end{Shaded}

\begin{verbatim}
## [1] FALSE
\end{verbatim}

\begin{Shaded}
\begin{Highlighting}[]
\CommentTok{# 其中一個敘述為 TRUE 則不管另一個是否為真都會回傳 TRUE}

\CommentTok{# 100 < 30 or 20 < 200 : TRUE}
\NormalTok{weight_a <}\StringTok{ }\DecValTok{30} \NormalTok{||}\StringTok{ }\NormalTok{weight_b <}\StringTok{ }\DecValTok{200}
\end{Highlighting}
\end{Shaded}

\begin{verbatim}
## [1] TRUE
\end{verbatim}

\subsubsection{3. 變數類型與資料型別 - Data Type}\label{---data-type}

在上述的舉例中我們並沒有明確告訴你們資料的\textbf{類型或型別(type)},因為我們可以一眼看出他是文字還是數值,但電腦並無法直接判斷,因此我們需要告訴他是什麼類型。然而現今許多程式語言都具有自動判斷的功能,因此有時候我們就會忽略掉變數的類型,但是變數的型別在某些時候相當的重要,因此我們需要認識一些基本的資料型別。

\paragraph{變數類型}

我們可以利用函式\texttt{class()}查詢變數的類型。

\begin{Shaded}
\begin{Highlighting}[]
\KeywordTok{class}\NormalTok{(}\StringTok{"偉豪"}\NormalTok{)}
\end{Highlighting}
\end{Shaded}

\begin{verbatim}
## [1] "character"
\end{verbatim}

\begin{Shaded}
\begin{Highlighting}[]
\KeywordTok{class}\NormalTok{(}\DecValTok{30}\NormalTok{)}
\end{Highlighting}
\end{Shaded}

\begin{verbatim}
## [1] "numeric"
\end{verbatim}

\begin{Shaded}
\begin{Highlighting}[]
\CommentTok{# 留意此處的 30 有加雙引號}
\KeywordTok{class}\NormalTok{(}\StringTok{"30"}\NormalTok{)}
\end{Highlighting}
\end{Shaded}

\begin{verbatim}
## [1] "character"
\end{verbatim}

\subparagraph{Numeric}\label{numeric}

在R語言中,任何整數或者小數(浮點數)在賦值後都會直接歸類為\texttt{numeric}

\begin{Shaded}
\begin{Highlighting}[]
\CommentTok{# 當我們輸入整數時}
\NormalTok{right_angle <-}\StringTok{ }\DecValTok{90}
\KeywordTok{class}\NormalTok{(right_angle)}
\end{Highlighting}
\end{Shaded}

\begin{verbatim}
## [1] "numeric"
\end{verbatim}

\begin{Shaded}
\begin{Highlighting}[]
\CommentTok{# 當我們輸入小數時}
\NormalTok{angle <-}\StringTok{ }\FloatTok{24.5}
\KeywordTok{class}\NormalTok{(angle)}
\end{Highlighting}
\end{Shaded}

\begin{verbatim}
## [1] "numeric"
\end{verbatim}

\emph{補充說明:在其他程式語言中,整數(integer)和浮點數(float)是兩種不同的資料型別,為什麽要區別主要原因是不同的資料型別會戰有不同的記憶體空間。}

\subparagraph{Integer}\label{integer}

\begin{quote}
在R語言中並不會常使用整數這個型別,主要原因是numeric已經有包含整數的運算,所以無須特別將\texttt{numeric}轉成\texttt{integer}。但在其他程式語言像:C,
C++, Java都會時常使用整數型別,因此特別介紹。
\end{quote}

\begin{Shaded}
\begin{Highlighting}[]
\CommentTok{# 要宣告成整數型態要在整數後面加上"L"}
\NormalTok{age <-}\StringTok{ }\NormalTok{24L}
\KeywordTok{class}\NormalTok{(age)}
\end{Highlighting}
\end{Shaded}

\begin{verbatim}
## [1] "integer"
\end{verbatim}


\end{document}
